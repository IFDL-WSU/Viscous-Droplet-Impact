\documentclass{article}
\usepackage[utf8]{inputenc}
\usepackage{amsmath}
\usepackage{amssymb}
\usepackage{enumerate}
\usepackage{enumitem}

\title{MATH 416 - Assignment 4}
\author{Rosemary Barrass }
\date{October 4, 2020}

\begin{document}

\setcounter{section}{4}
\setcounter{subsection}{2}
\subsection{}
Generate a random number $U$ \\
If $U < 0.35$ set $X=3$ and stop \\
If $U < 0.65$ set $X=1$ and stop \\
If $U < 0.85$ set $X=2$ and stop \\
Otherwise set $X=4$
\newpage

\setcounter{subsection}{12}
\subsection{}
One method to generate $X$ would be to use the following algorithm: \\
Step 1: Generate a random number $U$ \\
Step 2: d = $\sum_{j=0}^k e^{-\lambda}\lambda^j/j!$ \\
Step 3: $i=0,~p=e^{-\lambda}/d,~F=p$ \\
Step 4: If $U < F$, set $X=i$ and stop \\
Step 5: $p = \lambda p/d(i+1),~F = F+p, i = i+1$ \\
Step 6: Go to Step 4 \\

Another method to generate $X$ would be to use the following algorithm: \\
Step 1: $I=\text{Int}(\lambda)$ \\
Step 2: d = $\sum_{j=0}^k e^{-\lambda}\lambda^j/j!$ \\
Step 3: $i=0,~p=e^{-\lambda}/d,~F_I = p$ \\
Step 4: If I = i, go to Step 7\\
Step 5: $p = \lambda p/d(i+1),~F_I = F_I+p, i = i+1$ \\
Step 6: Go to Step 4 \\
Step 7: If $U \leq F$, go to Step 8. Else go to Step 10. \\
Step 8: $p = d(i+1)p/\lambda,~F = F-p, i = i-1$ \\
Step 9: If $U > F$, set $X = i+1$ and stop \\
Step 10: If $U < F$, set $X=i$ and stop \\
Step 11: $p = \lambda p/d(i+1),~F = F+p, i = i+1$ \\
Step 12: Go to Step 10 \\
\newpage

\setcounter{subsection}{16}
\subsection{}
Generate a random number $U$ \\
Step 1: $j=1,~p=\left(\frac{1}{2}\right)^{j+1} + \frac{\left(\frac{1}{2}\right)2^{j-1}}{3^j},~F=p$ \\
Step 2: If $U < F$, set $X=j$ and stop \\
Step 3: $j=j+1,~F=F+p$ \\
Step 4: Go to Step 2 \\
\newpage

\setcounter{section}{5}
\setcounter{subsection}{1}
\subsection{}
$F(x) = \int_2^x \frac{x-2}{2} dx$ if $2 \leq x \leq 3 = \frac{x^2}{4} - x + 1 = \frac{1}{4}(x-2)^2$ \\ 
$F(x) = \int_3^x \frac{2-\frac{x}{3}}{2} dx$ if $3 \leq x \leq 6 = x - \frac{x^2}{12} - \frac{9}{4} + \frac{1}{4}(3-2)^2 = -\frac{1}{12}(x-3)(x-9) +  \frac{1}{4}$ \\ 
$u = \frac{1}{4}(x-2)^2$ \\
$x = 2 \pm 2\sqrt{u}$ \\
We know that $X$ must be greater than $2$, so this will be \\
$x = 2 + 2\sqrt{u}$ \\
$u = -\frac{1}{12}(x-3)(x-9) +  \frac{1}{4}$
$x = 6 \pm 2\sqrt{3(1-u)}$ \\
We know that $X$ must be greater than $2$, so this will be \\
$x = 6 - 2\sqrt{3(1-u)}$ \\
Generate a random number $U_1$ \\
Generate a random number $U_2$ \\
If $U_1 < 0.25, X = 2 + 2\sqrt{U_2}$, stop \\
Otherwise, $X = 6 - 2\sqrt{3(1-U_2)}$, stop \\

\newpage
\setcounter{subsection}{7}
\subsection{}
\begin{enumerate}[label= (\alph*)]
    \item 
$F_1(x) = x$ \\
$u = x$ \\
$F_2(x) = x^3$ \\
$u = x^3$ \\
$x = u^{1/3}$
$F_3(x) = x^5$ \\
$u = x^5$ \\
$x = u^{1/5}$ \\
$p_1=p_2=p_3=\frac{1}{3}$ \\
Generate a random number $U_1$ \\
Generate a random number $U_2$ \\
If $U_1 < \frac{1}{3}$, $X = U_2$, stop \\
Else if $U_1 < \frac{2}{3}$, $X = U_2^{1/3}$, stop \\
Otherwise, $X = U_2^{1/5}$, stop \\
\end{enumerate}

\newpage
\setcounter{subsection}{14}
\subsection{}
One method to generate $X$ would be to use the following algorithm: \\
Using $g(x) = e^{-x},~ 0<x<\infty$, an exponentially distributed function with mean$=1$ for the rejection method:\\
$\frac{f(x)}{g(x)} = x$ \\
$\frac{d}{dx}\frac{f(x)}{g(x)} = 1$, $c=1$ \\
$\frac{f(x)}{cg(x)} = x$ \\
Step 1: Generate a random number $U_1$ and and set $Y = -\log U_1$ \\
Step 2: Generate a random number $U_2$ \\
Step 3: If $U_2 < Y$, set $X=Y$. Otherwise, return to Step 1. \\


Another method to generate $X$ would be to use the following algorithm: \\
Using $g(x) = \frac{1}{\sqrt{2\pi}}e^{-x^2/2},~ 0<x<\infty$, a normally distributed function with mean$=0$ and variance$=1$ for the rejection method:\\
$\frac{f(x)}{g(x)} = \sqrt{(2\pi)}*e^{\frac{x^2}{2}-x}$ \\
$\frac{d}{dx}\frac{f(x)}{g(x)} = e^{-x + x^2/2} \sqrt{2/\pi} (-1 + x)$, $c=1$ \\
$\frac{f(x)}{cg(x)} = e^{-x + x^2/2} \sqrt{2/\pi} (-1 + x)$ \\
Step 1: Generate $Y_1$, an exponential random variable with rate 1. \\
Step 2: Generate $Y_2$, an exponential random variable with rate 2. \\
Step 3: If $Y_2 - (Y_1-1)^2/2 > 0$, set $Y=Y_2-(Y_1-1)^2/2$ and go to Step 4. Otherwise, go to Step 1. \\
Step 4: Generate a random number $U_1$ and set $Z = Y_1$ if $U_1 \leq 1/2$ or $-Y_1$ if $U_1 > 1/2$. \\
Step 5: Generate a random number $U_2$ \\
Step 6: If $U_2 < e^{-Z + Z^2/2} \sqrt{2/\pi} (-1 + Z)$, set $X=Z$. Otherwise, return to Step 2. \\
The first method is the most efficient, as was proven in Example 5e in the textbook. It is clear that using a normally distributed random variable to generate a gamma distributed function adds more complicated and unnecessary steps.
\newpage
\setcounter{subsection}{17}
\subsection{}
Based on Example 5f:\\
Step 1: Generate $Y_1$, an exponential random variable with rate 1. \\
Step 2: Generate $Y_2$, an exponential random variable with rate 2. \\
Step 3: If $Y_2 - (Y_1-1)^2/2 > 0$, set $Y=Y_2-(Y_1-1)^2/2$ and go to Step 4. Otherwise, go to Step 1. \\
Step 4: Generate a random number $U$ and set $Z = Y_1$ if $U \leq 1/2$ or $-Y_1$ if $U > 1/2$. \\
\newpage
\subsection{}
\begin{enumerate}[label=(\alph*)]
    \item Inverse Transform Method \\
    $u = \frac{1}{2}(x+x^2)$ \\
    $x = \sqrt{\frac{1}{4}+2u} - \frac{1}{2}$ \\
    Step 1: Generate a random number $U$\\
    Step 2: $X = \sqrt{\frac{1}{4}+2U} - \frac{1}{2}$ \\
    \item Rejection Method \\
    $g(x) = -1,~0\leq x \leq 1$ \\
    $\frac{f(x)}{g(x)} = -\frac{1}{2}(x+x^2)$ \\
    $\frac{d}{dx}\frac{f(x)}{g(x)} = -\frac{1}{2}(1+2x) = -\frac{1}{2} - x$, $c = -\frac{1}{2}$ \\
    $\frac{f(x)}{cg(x)} = (x+x^2)$ \\
    Step 1: Generate random numbers $U_1,U_2$ \\
    Step 2: If $U_2 \leq U_1+U_1^2$, stop and set $X = U_1$. Otherwise, return to Step 1. \\
    \item The Composition Method \\
    $F_1(x) = x$ \\
$u = x$ \\
$F_2(x) = x^2$ \\
$u = x^2$ \\
$x = u^{1/2}$ \\
$p_1=p_2=\frac{1}{2}$ \\
Generate a random number $U_1$ \\
Generate a random number $U_2$ \\
If $U_1 < \frac{1}{2}$, $X = U_2$, stop \\
Otherwise, $X = U_2^{1/2}$, stop \\
\end{enumerate}
\end{document}
